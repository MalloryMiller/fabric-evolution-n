\documentclass[tikz,11pt,border=0.0pt]{standalone}

\usepackage{physics}
\usepackage{siunitx}
\usepackage{txfonts}

\usepackage{pgfplots}

\usetikzlibrary{arrows.meta}
%\usetikzlibrary{arrows.new}
\usetikzlibrary{patterns}
\usetikzlibrary{decorations.markings}
\usetikzlibrary{backgrounds}

\usepackage{xcolor}

\usetikzlibrary{fit}
\makeatletter
\tikzset{
  fitting node/.style={
    inner sep=0pt,
    fill=none,
    draw=none,
    reset transform,
    fit={(\pgf@pathminx,\pgf@pathminy) (\pgf@pathmaxx,\pgf@pathmaxy)}
  },
  reset transform/.code={\pgftransformreset}
}
\makeatother

\begin{document}


\definecolor{bgcolor}{HTML}{f7f6ee} % grayish
%\definecolor{bgcolor}{HTML}{ffffff}

\tikzset{every picture/.style={line cap=round, line width=0.75pt}}
\tikzset{
labeltitle/.style={font=\fontsize{13.25}{13.25}\selectfont},
typelabel/.style={labeltitle, font=\bf, black},
grpbox/.style={fill=bgcolor!280,draw=bgcolor!360}, % ,draw=bgcolor!360
%
mypicture/.style={scale=1,background rectangle/.style={fill=bgcolor}, inner frame xsep=2.0ex, inner frame ysep=2.0ex, show background rectangle},
}

\gdef\insetpanel#1#2#3#4{
	\gdef\xpadbox{0.95em}

	\fill[grpbox] (0em,#4-0.65em) rectangle ++({24.5em+\xpadbox},13.5em) node[fitting node] (R) {}; 
	\node[inner sep=0pt, anchor=south west] (ss) at (\xpadbox,#4) {\includegraphics[scale=1]{#1}};
	\node[yshift=+1.2em, anchor=west] at (R.north west) {#2};
	\node[labeltitle, yshift=-1.1em] at (R.north) {#3};
%	\node[labeltitle, yshift=-1em, xshift=4.2em] at (R.north east) {\large \textbf{Estimate}};
}

\begin{tikzpicture}[mypicture]
	
\gdef\yposDDM{-14.5em}

\gdef\titlediscrete{$\vu{Q}$ constructed from $\lbrace (\vb{b}_1,\vb{n}_1,\vb{v}_1), (\vb{b}_2,\vb{n}_2,\vb{v}_2), \cdots \rbrace$}
\gdef\titlecont{$\vu{Q}$ constructed from $b(\vu{r})$ and $n(\vu{r})$}

\insetpanel{DDM-simpleShear-Vi-15.pdf}{\Large \textit{(a)}\; Simple shear $(\gamma_{xz} = 1.8)$}{\titlediscrete}{0}
\insetpanel{SDM-simpleShear-Vi-15.pdf}{}{\titlecont}{\yposDDM}

\begin{scope}[yshift=-32em]
\insetpanel{DDM-axisymmetricCompression-Vi-15.pdf}{\Large \textit{(b)}\; Uniaxial compression $(\epsilon_{zz} = -0.73)$}{\titlediscrete}{0}
\insetpanel{SDM-axisymmetricCompression-Vi-15.pdf}{}{\titlecont}{\yposDDM}
\end{scope}
	
\end{tikzpicture}

\end{document}