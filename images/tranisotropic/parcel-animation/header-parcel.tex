% Nicholas Rathmann, 2023

\usepackage{tikz}
\usepackage{tikz-3dplot}

\tdplotsetmaincoords{65}{35} % 1

% For printing strain value to certain precision
\usepackage{pgffor}
\usepackage{pgfmath}
\usepackage{pgf}
\usepackage{xfp}
\pgfkeys{/pgf/number format/.cd, fixed, fixed zerofill, precision=2}

\usepackage{xcolor}
\definecolor{myred}{HTML}{a50f15}
\definecolor{mygray}{HTML}{d9d9d9}
\definecolor{bgcolor}{HTML}{f7f6ee}

%----------

\tikzset{
	every picture/.style={line cap=round, color=black, line width=0.35mm},
	boxtop/.style={ fill=mygray!80},
	boxend/.style={ fill=mygray!120},
	boxside/.style={fill=mygray!140},
}

%----------

\pgfmathsetmacro{\T}{1}  % e-folding time for uniaxial deformation with constant strain rate
\pgfmathsetmacro{\Nt}{100} % number of time steps
\pgfmathsetmacro{\straintargetUC}{-0.9} % strainzz target
\pgfmathsetmacro{\straintargetUE}{+2.0} % strainzz target

\tikzset{declare function = { 
	b(\t,\Te)  = exp(\t/\Te);
	Fz(\t,\Te) = b(\t,\Te)^(-1);
	Fh(\t,\Te) = b(\t,\Te)^(+0.5);
	fdt(\straintarget,\Te) = -\Te*ln(\straintarget+1)/\Nt;
}}

%----------

\newcommand{\parcelframe}[5]{%
	\begin{scope}[myred]
		\begin{scope}[dashed]
		\begin{scope}[canvas is yz plane at x=#1]
			\draw[] (#2,#3) -- (1+#2,#3) -- (1+#2,1+#3) -- (#2,1+#3) --cycle; % back left
		\end{scope}		
		\begin{scope}[canvas is yz plane at x={1+#1}]
			\draw[] (#2,#3) -- (1+#2,#3) -- (1+#2,1+#3) -- (#2,1+#3) --cycle; % front right
		\end{scope}
		\begin{scope}[canvas is xy plane at z=1+#3]
			\draw[] (#1,#2)   -- (1+#1,#2);
			\draw[] (#1,1+#2) -- (1+#1,1+#2);
		\end{scope}  
		\begin{scope}[canvas is xy plane at z=#3]
			\draw[] (#1,#2)   -- (1+#1,#2);
			\draw[] (#1,1+#2) -- (1+#1,1+#2);
		\end{scope}  		
		\end{scope}  
	\end{scope}  
}   

\newcommand{\parcel}[4]{%
	\draw[boxtop] (-#1/2,-#2/2,#3)-- (-#1/2,#2/2,#3) -- (#1/2,#2/2,#3) -- (#1/2,-#2/2,#3) --cycle; % top
	\draw[boxend] (#1/2,-#2/2,0)-- (#1/2,#2/2,0) -- (#1/2,#2/2,#3)  -- (#1/2,-#2/2,#3) --cycle; % right
	\draw[boxside] (-#1/2,-#2/2,0)-- (#1/2,-#2/2,0) -- (#1/2,-#2/2,#3)  -- (-#1/2,-#2/2,#3) --cycle; % front
	\draw[] (-#1/2,-#2/2,0)-- (-#1/2,#2/2,0) -- (#1/2,#2/2,0) -- (#1/2,-#2/2,0) --cycle; % behind bot
	\draw[] (-#1/2,#2/2,0) -- (-#1/2,#2/2,#3); % behind back
}   

\gdef\drawparcel#1#2#3{ 
	% ARGS: time step (n), strain_zz target, e-folding time scale (T)
	
	\pgfmathsetmacro{\dt}{fdt(#2,#3)}  % time step
	\pgfmathsetmacro\t{#1*\dt} % total time
	\pgfmathsetmacro\strainzz{Fz(\t,#3)-1} % strain_zz at time \t
	
	\pgfmathsetmacro\H{Fh(\t,#3)}
	\pgfmathsetmacro\Z{Fz(\t,#3)}		
	\parcel{\H}{\H}{\Z}{0}
	
	\pgfmathsetmacro\dh{-0.5}
	\pgfmathsetmacro\dz{0}
	\parcelframe{\dh}{\dh}{\dz}{}{}
	
	\node[anchor=west] at (-4.9em,+2.9em) {$\epsilon_{zz} = \pgfmathprintnumber{\fpeval{\strainzz}}$};
}